\documentclass[11pt]{article}
\usepackage{amsmath,amssymb,amsthm}
\usepackage{algorithm}
\usepackage[noend]{algpseudocode} 

%---enable russian----

\usepackage[utf8]{inputenc}
\usepackage[russian]{babel}

% PROBABILITY SYMBOLS
\newcommand*\PROB\Pr 
\DeclareMathOperator*{\EXPECT}{\mathbb{E}}


% Sets, Rngs, ets 
\newcommand{\N}{{{\mathbb N}}}
\newcommand{\Z}{{{\mathbb Z}}}
\newcommand{\R}{{{\mathbb R}}}
\newcommand{\Zp}{\ints_p} % Integers modulo p
\newcommand{\Zq}{\ints_q} % Integers modulo q
\newcommand{\Zn}{\ints_N} % Integers modulo N

% Landau 
\newcommand{\bigO}{\mathcal{O}}
\newcommand*{\OLandau}{\bigO}
\newcommand*{\WLandau}{\Omega}
\newcommand*{\xOLandau}{\widetilde{\OLandau}}
\newcommand*{\xWLandau}{\widetilde{\WLandau}}
\newcommand*{\TLandau}{\Theta}
\newcommand*{\xTLandau}{\widetilde{\TLandau}}
\newcommand{\smallo}{o} %technically, an omicron
\newcommand{\softO}{\widetilde{\bigO}}
\newcommand{\wLandau}{\omega}
\newcommand{\negl}{\mathrm{negl}} 

% Misc
\newcommand{\eps}{\varepsilon}
\newcommand{\inprod}[1]{\left\langle #1 \right\rangle}


\newcommand{\handout}[5]{
	\noindent
	\begin{center}
		\framebox{
			\vbox{
				\hbox to 5.78in { {\bf Научно-исследовательская практика} \hfill #2 }
				\vspace{4mm}
				\hbox to 5.78in { {\Large \hfill #5  \hfill} }
				\vspace{2mm}
				\hbox to 5.78in { {\em #3 \hfill #4} }
			}
		}
	\end{center}
	\vspace*{4mm}
}

\newcommand{\lecture}[4]{\handout{#1}{#2}{#3}{Scribe: #4}{Решение сравнений второй степени #1}}

\newtheorem{theorem}{Теорема}
\newtheorem{lemma}{Лемма}
\newtheorem{definition}{Определение}
\newtheorem{corollary}{Следствие}
\newtheorem{fact}{Факт}

% 1-inch margins
\topmargin 0pt
\advance \topmargin by -\headheight
\advance \topmargin by -\headsep
\textheight 8.9in
\oddsidemargin 0pt
\evensidemargin \oddsidemargin
\marginparwidth 0.5in
\textwidth 6.5in

\parindent 0in
\parskip 1.5ex

\begin{document}
	
	\lecture{}{Лето 2020}{}{Стахив Вадим}
	
	\section{Теория}
	Рассмотрим произвольное сравнение второй степени:	
	$ax^2+bx+c \equiv 0(mod p)$, где $a!\equiv 0(mod p)$
	т.к.$4a!\equiv 0(mod p)$, то умножая сравнение на $4a$, получаем эквивалентное сравнение:\\
	$4a^2x^2+4abx+4ac\equiv 0(mod p) \Leftrightarrow (2ax+b)^2\equiv b^2-4ac(mod p)$\\
	Оно сводится к сравнению вида:
	$y^2\equiv D(modp)$, где $D=b^2-4ac$ и $y=2ax+b$\\
	Отметим, что т.к. $2a$ и $p$ взаимно просты, то между $x$ и $y$ существует взаимно однозначное соответствие.\\
	Раз $D=b^2-4ac$, то сравнение имеет единственное решение при $D\equiv 0(mod p)$, имеет два решения, если $D$ квадратичный вычет по модулю p и не имеет решений, если $D$ квадратичный невычет по модулю $p$.
	
	\section{Алгоритм}
	Для решения сравнений второй степени необходимо решить следующее сравнение:\\
	$y^2\equiv D(modp)$, к которому сводится изначальное сраванение.\\
	Для решения этого сравнения я использую алгоритм Тонелли — Шенкса.
	\newpage
	\begin{algorithm}[ph]
		\caption{Алгоритм Тонелли — Шенкса}
		\label{alg:AlgName}
		\textbf{Входные данные:} $p$ - нечётное простое число, $a$ - целое число, являющееся квадратичным вычетом по модулю $p$\\
		\textbf{Выходные данные:} Вычет $R$, удовлетворяющий сравнению $R^2\equiv n(modp)$
		\begin{algorithmic}
			\State 1. Вычислить $a^{(p-1)/2}(modp)$. Результат должен быть равен 1 или -1 по малой теореме Ферма.
			\State 2. Если $a^{(p-1)/2}(modp) \equiv -1$ корней нет. Выход из алгоритма.
			\State 3. Если $a^{(p-1)/2}(modp) \equiv 1$ представить: $p-1=s*2^e$,где $s$ и $e$ положительны.
			\State 4. Найти число $n$, которое будет являться квадратным невычетом по модулю $p$
			\State 5. \textit{(Инициализация)}.\\ 
			\ \ \ $x\equiv a^{(s+1)/2}(modp)$ (Первое предположение о корне)\\
			\ \ \ $b\equiv a^s(modp)$ (Первое предположение о погрешности)\\
			\ \ \ $g\equiv n^s(modp)$ (Степень $g$ будет обновлять $x$ и $b$)\\
		    \ \ \ $r\equiv e(modp)$ ($e$ будет расти с каждым шагом алгоритма)
			\State 6. Имеем: $b^{2^{r-1}}=a^{s*2^{r-1}}=a^{(p-1)/2}\equiv 1(modp)$. Тогда существует такое наименьшее целое число $m$, что $0\le m\le r-1$ и $b^{2^m}\equiv1(modp)$
			\ \ \ Найти $m$ путем последовательных возведений в квадрат и сокращений по модулю $p$. Таким образом найдётся m такой, что $ord_p(b)=2^m$.
			\State 7. Если $m=0$, Вывести x.
			\State 8. Если $m>0$ изменяются следующие переменные: \\
			\ \ \ $x\equiv x*g^{2^{r-m-1}}(modp)$ \\
			\ \ \ $b\equiv b*g^{2^{r-m}}(modp)$ \\
			\ \ \ $g\equiv g^{2^{r-m}}(modp)$ \\
			\ \ \ $r\equiv m(modp)$
			\State 9. Вернитесь к шагу 6 с новыми значениями.
		\end{algorithmic}
	\end{algorithm}
	После нахождения решения сравнения $R$ второе решение сравнения находится как $p-R$. В самом начале я рассматриваю некоторые дополнительные тривиальные случаи для увеличения скорости алгоритма.
	\section{Машина}
	Компьютер, на котором проходили тестирования, имеет следующие основные характеристики:
	\begin{itemize}
		\item Процессор: \textbf{Intel(R) Core(TM) i7-7700HQ CPU @ 2.80 GHz}
			\item Видеоадаптер: \textbf{NVIDIA GeForce GTX 1050}
			\item Оперативная память: \textbf{16GB}
		\end{itemize}
		\section{Анализ}
		
		\begin{tabular}{|p{2cm}|p{2cm}|p{2cm}|p{2cm}|p{3.5cm}|p{3.5cm}|}
			\hline
			p & a & b & c & Время выполнения моего алгоритма & Время выполнения алгоритма Sage\\
			\hline
			$\approx 10^{100}$ &$\approx9^{78}$ &$\approx8^{67}$ & $\approx7^{32}$ & 2.81s & 0.39s\\
			$\approx10^{200}$ &$\approx9^{178}$ &$\approx8^{167}$ & $\approx7^{139}$ & 5.48s & 2.63s\\
			$\approx10^{300}$ & $\approx9^{200}$&$\approx8^{100}$ & $\approx7^{64}$ & 9.66s & 12.39s\\
			$\approx10^{400}$ &$\approx9^{378}$ &$\approx8^{267}$ & $\approx7^{239}$ & 15.32s & 32.9s\\
		    $\approx10^{450}$&$\approx9^{402}$ &$\approx8^{355}$ & $\approx7^{301}$ & 27.5s & 47.2s\\
			\hline
		\end{tabular}
		
		Тесты я проводил в онлайн компеляторе CoCalc. Исходя из результатов тестирования можно сделать вывод, что алгоритм sage для решения сравнений второй степени работает быстрее при небольших значениях коэффициентов и кольца, однако, он уступает моему алгоритму при достаточно больших числах.
		
		\paragraph{Bibliography.}
		\begin{itemize}
			\item Львовский С.М. - Набор и верстка в системе LATEX. 5-е изд. 2014.
			\item Wikipedia: Wikipedia, The Free Encyclopedia [online], http://www.wikipedia.org
			\item Square Roots from 1; 24, 51, 10 to Dan Shanks by Ezra Brown
			\item Zimmermann P., Casamayou A., Cohen N. et al. - Computational Mathematics with SageMath. 2018
			\item А.А.Илларионов - Теория чисел: Учебное пособие, 2016. 
		\end{itemize}
		
	\end{document}